% Options for packages loaded elsewhere
\PassOptionsToPackage{unicode}{hyperref}
\PassOptionsToPackage{hyphens}{url}
%
\documentclass[
]{article}
\usepackage{amsmath,amssymb}
\usepackage{lmodern}
\usepackage{ifxetex,ifluatex}
\ifnum 0\ifxetex 1\fi\ifluatex 1\fi=0 % if pdftex
  \usepackage[T1]{fontenc}
  \usepackage[utf8]{inputenc}
  \usepackage{textcomp} % provide euro and other symbols
\else % if luatex or xetex
  \usepackage{unicode-math}
  \defaultfontfeatures{Scale=MatchLowercase}
  \defaultfontfeatures[\rmfamily]{Ligatures=TeX,Scale=1}
\fi
% Use upquote if available, for straight quotes in verbatim environments
\IfFileExists{upquote.sty}{\usepackage{upquote}}{}
\IfFileExists{microtype.sty}{% use microtype if available
  \usepackage[]{microtype}
  \UseMicrotypeSet[protrusion]{basicmath} % disable protrusion for tt fonts
}{}
\makeatletter
\@ifundefined{KOMAClassName}{% if non-KOMA class
  \IfFileExists{parskip.sty}{%
    \usepackage{parskip}
  }{% else
    \setlength{\parindent}{0pt}
    \setlength{\parskip}{6pt plus 2pt minus 1pt}}
}{% if KOMA class
  \KOMAoptions{parskip=half}}
\makeatother
\usepackage{xcolor}
\IfFileExists{xurl.sty}{\usepackage{xurl}}{} % add URL line breaks if available
\IfFileExists{bookmark.sty}{\usepackage{bookmark}}{\usepackage{hyperref}}
\hypersetup{
  pdftitle={DA3 - Assignment 1},
  pdfauthor={Ersan Kucukoglu},
  hidelinks,
  pdfcreator={LaTeX via pandoc}}
\urlstyle{same} % disable monospaced font for URLs
\usepackage[margin=1in]{geometry}
\usepackage{graphicx}
\makeatletter
\def\maxwidth{\ifdim\Gin@nat@width>\linewidth\linewidth\else\Gin@nat@width\fi}
\def\maxheight{\ifdim\Gin@nat@height>\textheight\textheight\else\Gin@nat@height\fi}
\makeatother
% Scale images if necessary, so that they will not overflow the page
% margins by default, and it is still possible to overwrite the defaults
% using explicit options in \includegraphics[width, height, ...]{}
\setkeys{Gin}{width=\maxwidth,height=\maxheight,keepaspectratio}
% Set default figure placement to htbp
\makeatletter
\def\fps@figure{htbp}
\makeatother
\setlength{\emergencystretch}{3em} % prevent overfull lines
\providecommand{\tightlist}{%
  \setlength{\itemsep}{0pt}\setlength{\parskip}{0pt}}
\setcounter{secnumdepth}{5}
\usepackage{booktabs} \usepackage{longtable} \usepackage{array} \usepackage{multirow} \usepackage{wrapfig} \usepackage{float} \floatplacement{figure}{H}
\usepackage{booktabs}
\usepackage{longtable}
\usepackage{array}
\usepackage{multirow}
\usepackage{wrapfig}
\usepackage{float}
\usepackage{colortbl}
\usepackage{pdflscape}
\usepackage{tabu}
\usepackage{threeparttable}
\usepackage{threeparttablex}
\usepackage[normalem]{ulem}
\usepackage{makecell}
\usepackage{xcolor}
\usepackage{siunitx}
\newcolumntype{d}{S[input-symbols = ()]}
\ifluatex
  \usepackage{selnolig}  % disable illegal ligatures
\fi

\title{DA3 - Assignment 1}
\author{Ersan Kucukoglu}
\date{}

\begin{document}
\maketitle

In this Assignment, I analyzed Financial Specialists' wages from the
\href{https://osf.io/4ay9x/download}{CPS suvery}. First, as usual , I
started with data cleaning. I kept only the College Non-degree, BA, MA
graduated degrees, including employed-at-work people which is common in
the dataset. Since the age is mostly distributed between 22 and 64, I
filtered the age variable between 22 and 64. As we can see from Figure 1
that shows the Lowess vs.~quadratic specification with age, the
quadratic line just fit differently after 50 because quadratic tries to
force curvature, so I don't think it's relevant, the loess graph is the
closest fit on the data so its more reliable.Age has positive effect on
wage per hour until some point. I also filtered the Financial
Specialists who work as full-time employees (hours \textgreater= 40). I
created the wage per hour variable by dividing weekly earnings by hours.
I only focused on being employed at work because the other earnings are
difficult to measure like self-earning and included those who reported
20 hours or more as their usual weekly time worked. As a result, I have
Financial Specialists data with 2437 observations which have only
College, BA, MA, and as the highest education level.

\emph{Four Regression Models}

\begin{itemize}
\tightlist
\item
  Model 1 : wage\_per\_hour \textasciitilde{} age + agesq
\item
  Model 2 : wage\_per\_hour \textasciitilde{} age + agesq + male +
  female
\item
  Model 3 : wage\_per\_hour \textasciitilde{} age + agesq + male +
  female + College + BA+ MA
\item
  Model 4 : wage\_per\_hour \textasciitilde{} age + agesq + male +
  female + College + BA+ MA + female* College + female* BA + female*MA
  +Private + Government
\end{itemize}

Using the data, four prediction models were built which can be seen
above. I started by adding age and gender and then gradually added more
variables. Table 1 shows the regression coefficients of the four
regression models. To find the best model, i first evaluated the BIC
values along with R-squared and the RMSE in Table 1. Second, by using
k-fold cross validation I set k=4, it means splitting the data into four
in a random fashion to define the four test sets. According to the both
approaches, Model 3 and Model 4 have the best prediction properties.
They have the lowest BIC (19,833.6 and 19,854.7), and also they have the
lowest average cross- validated RMSE values(14.054 and 14.097), in the
Table 2. In addition to the Table 1, we can see the number of the
variables and averaged RMSE on the test samples from the Prediction
Performance and model complexity graph. Model performance is better as
number of predictor variables is larger from the beginning. However,
after a certain point (6), model performance is worse, as number of
predictor variables is getting larger. According to performance
measures, the actual difference between two models is very small.
Choosing a simple model can be valuable as it may help us avoid
overfitting the live data. Since the Model 3 and Model 4 have RMSE
values that are very close, it make sense to choose Model 3.

\pagebreak

\hypertarget{appendix}{%
\subsection{Appendix}\label{appendix}}

\begingroup\fontsize{7}{9}\selectfont

\begin{longtable}[t]{lllll}
\caption{\label{tab:unnamed-chunk-10} Regression Models for predicting earning per hour}\\
\toprule
  & reg1 & reg2 & reg3 & reg4\\
\midrule
Dependent Var.: & wage\_per\_hour & wage\_per\_hour & wage\_per\_hour & wage\_per\_hour\\
 &  &  &  & \\
(Intercept) & -9.699* (3.850) & -15.84*** (3.742) & -10.59** (3.785) & -11.26** (3.922)\\
age & 1.754*** (0.2024) & 1.880*** (0.1948) & 1.801*** (0.1927) & 1.805*** (0.1933)\\
agesq & -0.0168*** (0.0025) & -0.0181*** (0.0024) & -0.0169*** (0.0023) & -0.0170*** (0.0024)\\
\addlinespace
male &  & 7.091*** (0.5875) & 6.044*** (0.5892) & 6.824*** (1.329)\\
College &  &  & -10.22*** (0.9760) & -12.46*** (1.693)\\
BA &  &  & -3.491*** (0.7519) & -3.791*** (1.022)\\
Private &  &  &  & 0.2075 (0.8379)\\
female x College &  &  &  & 3.266 (2.122)\\
\addlinespace
female x BA &  &  &  & 0.6946 (1.510)\\
\_\_\_\_\_\_\_\_\_\_\_\_\_\_\_\_ & \_\_\_\_\_\_\_\_\_\_\_\_\_\_\_\_\_\_\_ & \_\_\_\_\_\_\_\_\_\_\_\_\_\_\_\_\_\_\_ & \_\_\_\_\_\_\_\_\_\_\_\_\_\_\_\_\_\_\_ & \_\_\_\_\_\_\_\_\_\_\_\_\_\_\_\_\_\_\_\\
S.E. type & Heteroskedast.-rob. & Heteroskedast.-rob. & Heteroskedast.-rob. & Heteroskedast.-rob.\\
AIC & 20,031.9 & 19,890.2 & 19,798.8 & 19,802.6\\
BIC & 20,049.3 & 19,913.4 & 19,833.6 & 19,854.7\\
\addlinespace
RMSE & 14.728 & 14.301 & 14.023 & 14.017\\
R2 & 0.08064 & 0.13328 & 0.16656 & 0.16732\\
Observations & 2,437 & 2,437 & 2,437 & 2,437\\
No. Variables & 2 & 3 & 5 & 8\\
\bottomrule
\end{longtable}
\endgroup{}

\begingroup\fontsize{10}{12}\selectfont

\begin{longtable}[t]{lrrrr}
\caption{\label{tab:unnamed-chunk-11} 4-fold cross-validation and RMSE}\\
\toprule
Resample & Model1 & Model2 & Model3 & Model4\\
\midrule
Fold1 & 14.48274 & 13.93398 & 13.57886 & 13.56791\\
Fold2 & 14.48202 & 14.33291 & 14.02649 & 14.12810\\
Fold3 & 15.01655 & 14.46690 & 14.28230 & 14.27980\\
Fold4 & 15.01504 & 14.61914 & 14.39369 & 14.40076\\
Average & 14.75150 & 14.34049 & 14.07383 & 14.09775\\
\bottomrule
\end{longtable}
\endgroup{}

\includegraphics{Assignment_1_files/figure-latex/unnamed-chunk-12-1.pdf}

\includegraphics{Assignment_1_files/figure-latex/unnamed-chunk-13-1.pdf}

\end{document}
